\documentclass[12pt, a4paper]{article}
\usepackage[utf8]{inputenc}
\usepackage{ragged2e}

\usepackage{graphicx, geometry, hyperref, wrapfig, amsmath, subcaption}
\usepackage[dvipsnames]{xcolor}

\definecolor{silver}{RGB}{200,200,200}
\hypersetup{colorlinks=true, linkcolor=RoyalBlue, urlcolor=RoyalBlue}

 \geometry{
 a4paper,
 total={175mm,257mm},
 left=20mm,
 top=15mm,
 }

\usepackage{xcolor} % for defining colour
\usepackage{titlesec} % for customizing sections

% \usepackage{times}        % Use Times New Roman font
% \usepackage{helvet}       % Use Helvetica font
\usepackage{palatino}

\usepackage[T1]{fontenc}

\setlength\parindent{0pt}

% %%%%%%%%%%%%%%%%%%%%%%%%%%%%%%%%%%%%%%%%%%%%%%%%%%%%%%%%%%%%%%
\titleformat{\section}
{\color{UM_DarkBlue}\normalfont\large\bfseries}
{\color{UM_DarkBlue}\thesection}{1em}{}

%%%%%%%%%%%%%%%%%%%%%%%%%%%%%%%%%%%%%%%%%%%%%%%%%%%%%%%%%%%%%%%
\definecolor{UM_Brown}{HTML}{3D190D}
\definecolor{UM_DarkBlue}{HTML}{2264B0}
\definecolor{UM_LightBlue}{HTML}{1CA9E1}
\definecolor{UM_Orange}{HTML}{fEB415}

%%%%%%%%%%%%%%%%%%%%%%%%%%%%%%%%%%%%%%%%%%%%%%%%%%%%%%%%%%%%%%%%

\newcommand{\eg}{{\it e.g.}}
\newcommand{\ie}{{\it i.e.}}

% %%%%%%%%%%%%%%%%%%%%%%%%%%%%%%%%%%%%%%%%%%%%%%%%%%%%%%%%%%%%%%
% \hypersetup{
%     draft=false,
%     final=true,
%     colorlinks=true,
%     citecolor=UM_DarkBlue,
%     anchorcolor=yellow,
%     linkcolor=UM_DarkBlue,
%     urlcolor=UM_DarkBlue,
%     filecolor=green,      
%     pdfpagemode=FullScreen,
%     bookmarksopen=false
%     }
\usepackage{amsmath,amsfonts,amssymb,bm}

%%%%%%%%%%%%%%%%%%%%%%%%%%%%%%%%%%%%%%%%%%%%%%%%%%%%%%%%%%%%%
% Sets and Notations
\newcommand{\reals}{\mathbb{R}}
\newcommand{\integers}{\mathbb{Z}}

%%%%%%%%%%%%%%%%%%%%%%%%%%%%%%%%%%%%%%%%%%%%%%%%%%%%%%%%%%%%%
% Vectors and Matrices
% \renewcommand{\vec}[1]{\bm{\mathrm{#1}}}
\newcommand{\dotp}{\,\boldsymbol{\cdot}\,}
\newcommand{\grad}[1]{\vec{\nabla}#1}
\renewcommand{\div}[1]{\vec{\nabla}\!\dotp\!\vec{#1}}
\newcommand{\curl}[1]{\vec{\nabla}\!\times\!\vec{#1}}



%%%%%%%%%%%%%%%%%%%%%%%%%%%%%%%%%%%%%%%%%%%%%%%%%%%%%%%%%%%%%
% Derivatives
\newcommand{\dv}[2]{\frac{d#1}{d#2}}
\newcommand{\ndv}[3][2]{\frac{d^{\,#1}#2}{d#3^{\,#1}}}

\newcommand{\pdv}[2]{\frac{\partial#1}{\partial#2}}
\newcommand{\npdv}[3][2]{\frac{\partial^{\,#1}#2}{\partial#3^{#1}}}
 
\title{OWO-GAship}
\author{Anik Mandal}
\date{January 2025}
\pagenumbering{arabic}

%====================================================================================================
\begin{document}

\begin{minipage}[t][][c]{0.1\textwidth}
    \begin{flushleft}
        \includegraphics[height=2cm]{tex-resources/Ashoka Logo.png}
    \end{flushleft}
\end{minipage}
\begin{minipage}[t][][c]{0.85\textwidth}
    \begin{center}
        {\LARGE Oscillations, Wave and Optics}\\ \vspace{0.5em}
        \textsc{(Spring 2025)}\\
        \vspace{1em}
        \textbf{\Large ASSIGNMENT-3} \\
    \end{center}
\end{minipage}
\vspace{10pt}\\
\rule[0em]{\textwidth}{0.75pt}

\flushleft{Topics: Longitudinal Standing Waves, Traveling Waves \hfill
Total : 40 \\
\flushleft{Date: 3rd Mar, 2025}\hfill
\fbox{\textbf{\large 
Due: 16th Mar, 2025} (EoD)}\\
\vspace{.2cm}
\rule[0em]{\textwidth}{1.75pt}
\vspace{-1cm}

%====================================================================================================
%====================================================================================================
\justifying

\section*{Part:A | Longitudinal Standing Waves \hfill \textbf{[15]}}
\noindent
\textbf{(1)}  A simple model of an ionic crystal consists of a linear array of a great many 
equally-spaced atoms of alternating masses $M$ and $m$, where $m<M$. The masses are connected by 
identical chemical bonds that are modeled as springs of spring constant $K$.\\

\textbf{i.} Show that the frequencies of the system's longitudinal modes of vibration either lie in 
the band 0 to $(2\,K/M)^{1/2}$ or in the band $(2\,K/m)^{1/2}$ to $[2\,K\,(1/M+1/m)]^{1/2}$.\\

\textbf{ii.} Show that, in the long-wavelength limit, modes whose frequencies lie in the lower band 
are such that neighboring atomics move in the same direction, whereas modes whose frequencies lie 
in the upper band are such that neighboring atoms move in opposite directions. The lower band is 
known as the acoustic branch, whereas the upper band is known as the optical branch.
\hfill \textbf{(10 + 5)}

\section*{Part:B | Traveling Waves \hfill \textbf{[25]}}
\textbf{(1)} Demonstrate that for a transverse traveling wave propagating on a stretched string,
\begin{align*}
    \langle {\cal I}\rangle = v\,\langle {\cal E}\rangle
\end{align*}
where $\langle {\cal I}\rangle$ is the mean energy flux along the string due to the wave, 
$\langle {\cal E}\rangle$ is the mean wave energy per unit length, and $v$ is the phase 
velocity of the wave.\hfill \textbf{3}\\

\noindent
\textbf{(2)} A lossy transmission line has a resistance per unit length ${\cal R}$,
in addition to an inductance per unit length ${\cal L}$, and a capacitance per unit length 
${\cal C}$. The resistance can be considered to be in series with the inductance.

\textbf{i.} Demonstrate that the Telegrapher's equations generalize to,
\begin{align*}
    \frac{\partial V}{\partial t} &=-\frac{1}{\cal C}\frac{\partial I}{\partial x}\\  
    \frac{\partial I}{\partial t} &=-\frac{\cal R}{\cal L} I - \frac{1}{\cal L}\frac{\partial V}{\partial x}
\end{align*}

\textbf{ii.} Derive an energy conservation equation of the form
\begin{align*}
    \frac{\partial{\cal E}}{\partial t} + \frac{\partial {\cal I}}{\partial x} =- {\cal R}I^{2}
\end{align*}
where ${\cal E}$ is the energy per unit length along the line, and ${\cal I}$ the energy flux. 
Give expressions for ${\cal E}$ and ${\cal I}$. What does the right-hand side of the previous 
equation represent?

\textbf{iii.}Show that the current obeys the wave-diffusion equation
\begin{align*}
    \frac{\partial^{2} I}{\partial t^{2}}+ \frac{\cal R}{\cal L}\frac{\partial I}{\partial t} = \frac{1}{{\cal L}{\cal C}}\frac{\partial^{2} I}{\partial x^{2}}
\end{align*}

\textbf{iv.} Consider the low-resistance, high-frequency, limit $\omega\gg {\cal R}/{\cal L}$.
Demonstrate that a signal propagating down the line varies as
\begin{align*}
    I(x,t) &\simeq I_0\cos[k(vt-x)]{\rm e}^{-x/\delta}\\	   
    V(x,t) &\simeq Z I_0\cos[k(vt-x)-1/(k\delta)]{\rm e}^{-x/\delta}
\end{align*}
where $k=\omega/v$, $v=1/\sqrt{{\cal L}\,{\cal C}}$, $\delta = 2\,Z/{\cal R}$, 
and $Z=\sqrt{{\cal L}/{\cal C}}$. Show that $k\,\delta \gg 1$; that is, 
the decay length of the signal is much longer than its wavelength. 
Estimate the maximum useful length of a low-resistance, high-frequency, lossy transmission line.
\hfill\textbf{5+3+3+8}\\

\noindent
\textbf{(3)} At normal incidence, the mean radiant power from the Sun illuminating one square 
meter of the Earth's surface is $1.35$ kW. Show that the peak amplitude of the electric component 
of solar electromagnetic radiation at the Earth's surface is $1010{\rm V}{\rm m}^{-1}$. 
Demonstrate that the corresponding peak amplitude of the magnetic component is 
$2.7{\rm A}{\rm m}^{-1}$. \hfill \textbf{3}


\end{document}