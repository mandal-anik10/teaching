\documentclass[12pt, a4paper]{article}
\usepackage[utf8]{inputenc}
\usepackage{ragged2e}

\usepackage{graphicx, geometry, hyperref, wrapfig}
\usepackage[dvipsnames]{xcolor}

\definecolor{silver}{RGB}{200,200,200}
\hypersetup{colorlinks=true, linkcolor=RoyalBlue, urlcolor=RoyalBlue}

 \geometry{
 a4paper,
 total={175mm,257mm},
 left=20mm,
 top=15mm,
 }

\usepackage{xcolor} % for defining colour
\usepackage{titlesec} % for customizing sections

% \usepackage{times}        % Use Times New Roman font
% \usepackage{helvet}       % Use Helvetica font
\usepackage{palatino}

\usepackage[T1]{fontenc}

\setlength\parindent{0pt}

% %%%%%%%%%%%%%%%%%%%%%%%%%%%%%%%%%%%%%%%%%%%%%%%%%%%%%%%%%%%%%%
\titleformat{\section}
{\color{UM_DarkBlue}\normalfont\large\bfseries}
{\color{UM_DarkBlue}\thesection}{1em}{}

%%%%%%%%%%%%%%%%%%%%%%%%%%%%%%%%%%%%%%%%%%%%%%%%%%%%%%%%%%%%%%%
\definecolor{UM_Brown}{HTML}{3D190D}
\definecolor{UM_DarkBlue}{HTML}{2264B0}
\definecolor{UM_LightBlue}{HTML}{1CA9E1}
\definecolor{UM_Orange}{HTML}{fEB415}

%%%%%%%%%%%%%%%%%%%%%%%%%%%%%%%%%%%%%%%%%%%%%%%%%%%%%%%%%%%%%%%%

\newcommand{\eg}{{\it e.g.}}
\newcommand{\ie}{{\it i.e.}}

% %%%%%%%%%%%%%%%%%%%%%%%%%%%%%%%%%%%%%%%%%%%%%%%%%%%%%%%%%%%%%%
% \hypersetup{
%     draft=false,
%     final=true,
%     colorlinks=true,
%     citecolor=UM_DarkBlue,
%     anchorcolor=yellow,
%     linkcolor=UM_DarkBlue,
%     urlcolor=UM_DarkBlue,
%     filecolor=green,      
%     pdfpagemode=FullScreen,
%     bookmarksopen=false
%     }
\usepackage{amsmath,amsfonts,amssymb,bm}

%%%%%%%%%%%%%%%%%%%%%%%%%%%%%%%%%%%%%%%%%%%%%%%%%%%%%%%%%%%%%
% Sets and Notations
\newcommand{\reals}{\mathbb{R}}
\newcommand{\integers}{\mathbb{Z}}

%%%%%%%%%%%%%%%%%%%%%%%%%%%%%%%%%%%%%%%%%%%%%%%%%%%%%%%%%%%%%
% Vectors and Matrices
% \renewcommand{\vec}[1]{\bm{\mathrm{#1}}}
\newcommand{\dotp}{\,\boldsymbol{\cdot}\,}
\newcommand{\grad}[1]{\vec{\nabla}#1}
\renewcommand{\div}[1]{\vec{\nabla}\!\dotp\!\vec{#1}}
\newcommand{\curl}[1]{\vec{\nabla}\!\times\!\vec{#1}}



%%%%%%%%%%%%%%%%%%%%%%%%%%%%%%%%%%%%%%%%%%%%%%%%%%%%%%%%%%%%%
% Derivatives
\newcommand{\dv}[2]{\frac{d#1}{d#2}}
\newcommand{\ndv}[3][2]{\frac{d^{\,#1}#2}{d#3^{\,#1}}}

\newcommand{\pdv}[2]{\frac{\partial#1}{\partial#2}}
\newcommand{\npdv}[3][2]{\frac{\partial^{\,#1}#2}{\partial#3^{#1}}}
 
\title{OWO-GAship}
\author{Anik Mandal}
\date{January 2025}
\pagenumbering{arabic}

%====================================================================================================
\begin{document}

\begin{minipage}[t][][c]{0.1\textwidth}
    \begin{flushleft}
        \includegraphics[height=2.5cm]{tex-resources/Ashoka Logo.png}
    \end{flushleft}
\end{minipage}
\begin{minipage}[t][][c]{0.85\textwidth}
    \begin{center}
        {\LARGE Oscillations, Wave and Optics}\\ \vspace{0.5em}
        \textsc{(Spring 2025)}\\
        \vspace{1em}
        \textbf{\Large ASSIGNMENT-4} \\
    \end{center}
\end{minipage}
\vspace{10pt}\\
\rule[0em]{\textwidth}{0.75pt}

\flushleft{Topics: MATHS | Fourier Returns}\hfill 
Total Marks: 50   \\
\flushleft{Date: 1st Apr, 2025}\hfill
\fbox{\textbf{\large 
Due: 30th Apr, 2025} (EoD) No extension!}\\
\vspace{.2cm}
\rule[0em]{\textwidth}{1.75pt}
\vspace{-1cm}
%====================================================================================================
%====================================================================================================
\justifying

\section*{Problem-1: \hfill \textbf{[30]}}
\begin{itemize}
    \item Make a hand-drawn plot of the given functions.
    \item Find fourier series of the given functions.
    \item Plot the truncated fouier series (upto certain terms) using graphing calculators 
    or programming.
\end{itemize}

\textbf{(1)} Sawtooth wave, \[ f(x) =
\begin{cases} 
x, & \text{for } 0\leq x < \pi \\
x-2\pi, & \text{for } \pi\leq x <2\pi
\end{cases}
\] 

\textbf{(2)} Reverse sawtooth wave, \[ f(x) =
\begin{cases} 
-\frac{1}{2}(\pi+x), & \text{for } -\pi\leq x < 0 \\
\frac{1}{2}(\pi-x), & \text{for } 0\leq x <\pi
\end{cases}
\] 

\textbf{(3)} Triangular wave, \[ f(x) =
\begin{cases} 
-x, & \text{for } -\pi\leq x < 0 \\
x, & \text{for } 0\leq x <\pi
\end{cases}
\] 

\textbf{(4)} \[ f(x) =
\begin{cases} 
4x(1+x), & \text{for } -1\leq x < 0 \\
4x(1-x), & \text{for } 0\leq x <1
\end{cases}
\] 

\textbf{(5)} Full-wave rectifier, \[ f(x) =
\begin{cases} 
\sin(\omega t), & \text{for } 0 \leq x < \pi/ \omega \\
\sin(\omega t), & \text{for } -\pi/\omega \leq x < 0
\end{cases}
\] 

\textbf{(6)} Rectangular pulse ($n<\pi$)\[ \delta_n(x) =
\begin{cases} 
n, & \text{for } |x| < \frac{1}{2n} \\
0, & \text{for } \pi > |x| > \frac{1}{2n}
\end{cases}
\] 

\section*{Problem-2: (Arfken 19.2.17) \hfill \textbf{[10]}}

\textbf{(1)} Show that the Dirac delta function $\delta(x-a)$, expanded in a Fourier sine series in
the half-interval $(0, L)$ $(0 < a < L)$ is given by,
\begin{align*}
    \delta(x-a) = \frac{1}{2} \sum_{n=1}^{\infty} sin(\frac{n\pi a}{L})sin(\frac{n\pi x}{L})
\end{align*}
Note that this series actually describes $-\delta(x + a) + \delta(x-a)$ in the interval
$(-L , L)$.\\
\textbf{(2)} By integrating both sides of the preceding equation from 0 to x, show that the
cosine expansion of the square wave
\[ f(x) =
\begin{cases} 
0, & \text{for } 0 \leq x < a \\
1, & \text{for } a \leq x < L,
\end{cases}
\] 
is,
\[
f(x) = \frac{2}{\pi} \sum_{n=1}^{\infty} \frac{1}{n} \sin\left(\frac{n\pi a}{L}\right) 
- \frac{2}{\pi} \sum_{n=1}^{\infty} \frac{1}{n} \sin\left(\frac{n\pi a}{L}\right) 
\cos\left(\frac{n\pi x}{L}\right),
\]
for $0 \leq x < L$.\\
\textbf{(3)} Show that the term $\frac{2}{\pi}\sum_{n=1}^{\infty}\frac{1}{n}\sin(\frac{n\pi a}{l})$ 
is the average of f(x) on (0, L)
\hfill\textbf{(3+4+3)}

\section*{Problem-3: (Arfken 19.2.20) \hfill \textbf{[10]}}
\textbf{(1)} A string, clamped at x = 0 and at x = L, is vibrating freely. Its motion is described 
by the wave equation
\begin{align*}
    \frac{\partial^2 u(x,t)}{\partial t^2} = v^2 \frac{\partial^2 u(x,t)}{\partial x^2}
\end{align*}
Assume a Fourier expansion of the form,
\[
u(x, t) = \sum_{n=1}^{\infty} b_n(t) \sin\left(\frac{n\pi x}{L}\right)
\]
and determine the coefficients $b_n(t)$. The initial conditions are
\[
u(x, 0) = f(x)\ and\ \frac{\partial }{\partial t}u(x, 0) = g(x)
\]
Note (but don't care about this now), This is only half the conventional Fourier orthogonality integral interval. However, as long 
as only the sines are included here, the Sturm-Liouville boundary conditions are still satisfied 
and the functions are orthogonal.\\
\textbf{(2)} We assume now that the presence of a resisting medium will damp the vibrations 
according to the equation
\begin{align*}
    \frac{\partial^2 u(x,t)}{\partial t^2} = v^2 \frac{\partial^2 u(x,t)}{\partial x^2} - k \frac{\partial u(x,t)}{\partial t}
\end{align*}
Introduce a Fourier expansion similar to above form.
Again determine the coefficients $b_n(t)$. Take the initial and boundary conditions to be the same 
as above. Assume the damping to be small $(k^2 < \frac{4n\pi v}{L})$.\\
\textbf{(3)} Repeat, but assume the damping to be large $(k^2 > \frac{4n\pi v}{L})$. \hfill\textbf{4+3+3}
\end{document}
