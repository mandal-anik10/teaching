\documentclass[12pt, a4paper]{article}
\usepackage[utf8]{inputenc}

\usepackage{graphicx, geometry, hyperref, wrapfig}
\usepackage[dvipsnames]{xcolor}

\definecolor{silver}{RGB}{200,200,200}
\hypersetup{colorlinks=true, linkcolor=RoyalBlue, urlcolor=RoyalBlue}

 \geometry{
 a4paper,
 total={175mm,257mm},
 left=20mm,
 top=15mm,
 }

\usepackage{xcolor} % for defining colour
\usepackage{titlesec} % for customizing sections

% \usepackage{times}        % Use Times New Roman font
% \usepackage{helvet}       % Use Helvetica font
\usepackage{palatino}

\usepackage[T1]{fontenc}

\setlength\parindent{0pt}

% %%%%%%%%%%%%%%%%%%%%%%%%%%%%%%%%%%%%%%%%%%%%%%%%%%%%%%%%%%%%%%
\titleformat{\section}
{\color{UM_DarkBlue}\normalfont\large\bfseries}
{\color{UM_DarkBlue}\thesection}{1em}{}

%%%%%%%%%%%%%%%%%%%%%%%%%%%%%%%%%%%%%%%%%%%%%%%%%%%%%%%%%%%%%%%
\definecolor{UM_Brown}{HTML}{3D190D}
\definecolor{UM_DarkBlue}{HTML}{2264B0}
\definecolor{UM_LightBlue}{HTML}{1CA9E1}
\definecolor{UM_Orange}{HTML}{fEB415}

%%%%%%%%%%%%%%%%%%%%%%%%%%%%%%%%%%%%%%%%%%%%%%%%%%%%%%%%%%%%%%%%

\newcommand{\eg}{{\it e.g.}}
\newcommand{\ie}{{\it i.e.}}

% %%%%%%%%%%%%%%%%%%%%%%%%%%%%%%%%%%%%%%%%%%%%%%%%%%%%%%%%%%%%%%
% \hypersetup{
%     draft=false,
%     final=true,
%     colorlinks=true,
%     citecolor=UM_DarkBlue,
%     anchorcolor=yellow,
%     linkcolor=UM_DarkBlue,
%     urlcolor=UM_DarkBlue,
%     filecolor=green,      
%     pdfpagemode=FullScreen,
%     bookmarksopen=false
%     }
\usepackage{amsmath,amsfonts,amssymb,bm}

%%%%%%%%%%%%%%%%%%%%%%%%%%%%%%%%%%%%%%%%%%%%%%%%%%%%%%%%%%%%%
% Sets and Notations
\newcommand{\reals}{\mathbb{R}}
\newcommand{\integers}{\mathbb{Z}}

%%%%%%%%%%%%%%%%%%%%%%%%%%%%%%%%%%%%%%%%%%%%%%%%%%%%%%%%%%%%%
% Vectors and Matrices
% \renewcommand{\vec}[1]{\bm{\mathrm{#1}}}
\newcommand{\dotp}{\,\boldsymbol{\cdot}\,}
\newcommand{\grad}[1]{\vec{\nabla}#1}
\renewcommand{\div}[1]{\vec{\nabla}\!\dotp\!\vec{#1}}
\newcommand{\curl}[1]{\vec{\nabla}\!\times\!\vec{#1}}



%%%%%%%%%%%%%%%%%%%%%%%%%%%%%%%%%%%%%%%%%%%%%%%%%%%%%%%%%%%%%
% Derivatives
\newcommand{\dv}[2]{\frac{d#1}{d#2}}
\newcommand{\ndv}[3][2]{\frac{d^{\,#1}#2}{d#3^{\,#1}}}

\newcommand{\pdv}[2]{\frac{\partial#1}{\partial#2}}
\newcommand{\npdv}[3][2]{\frac{\partial^{\,#1}#2}{\partial#3^{#1}}}
 
\title{OWO-GAship}
\author{Anik Mandal}
\date{January 2025}
\pagenumbering{arabic}

%====================================================================================================
\begin{document}

\begin{minipage}[t][][c]{0.1\textwidth}
    \begin{flushleft}
        \includegraphics[height=2cm]{tex-resources/Ashoka Logo.png}
    \end{flushleft}
\end{minipage}
\begin{minipage}[t][][c]{0.85\textwidth}
    \begin{center}
        {\LARGE Oscillations, Wave and Optics}\\ \vspace{0.5em}
        \textsc{(Spring 2025)}\\
        \vspace{1em}
        \textbf{\Large ASSIGNMENT-0} \\
    \end{center}
\end{minipage}
\vspace{10pt}\\
\rule[0em]{\textwidth}{0.75pt}

\flushleft{Topics: Basic Math (Calculus, Taylor Series, Fourier Series, ODE)}\hfill 
Total Marks: 50   \\
\flushleft{Date: 22nd Jan, 2024}\hfill
\fbox{\textbf{\large 
Due: 23rd Feb, 2024} (EoD)}\\
\vspace{.2cm}
\rule[0em]{\textwidth}{1.75pt}
\vspace{-1cm}
%====================================================================================================
%====================================================================================================

\section*{Problem-1 \hfill \textbf{[10]}}
\textbf{(a)} Solve $4y'' + 4y' + 37y=0$ and find y(x) for the given boundary conditions: \\
(i) $y(x=0)=0$,(ii) $y(x=\frac{\pi}{6})=exp(-\frac{\pi^2}{12})$.\\
Crosscheck the solution; check whether your solution satisfies the ODE.\hfill \textbf{3+2}\\

\textbf{(b)} Make a hand-drawn plot of the solution in the x-y plane using the reference informations:\\
(i) $exp(-\frac{1}{2})\approx 0.6$, (ii) $exp(-\frac{5}{4})\approx 0.3$, (iii) $exp(-\frac{9}{4})\approx 0.1$\\
Also briefly mention how you are using the provided informations for plotting. \hfill\textbf{2+1}\\

\textbf{(c)} Taylor expand the solution about x=0. (\textit{Error/deviation of the order $x^5$is acceptable}).\\
Also, estimate the leading order error term at x=1. \hfill \textbf{2}

%====================================================================================================
\section*{Problem-2 \hfill \textbf{[10]}}
\textbf{(a)} (i) $x=sint$, (ii) $y=cos2t$\\ 
plot these two equations in the t-x and t-y planes, respectively. (\textit{In range $t=[0, 2\pi]$}).\hfill \textbf{2+2}\\
\textbf{(b)} For a constant t, you will get the x-value and y-value using those two equations.
Use a set of t-values to get a set of x-values and y-values. 
Use those x-values and y-values to find the trajectory of the particle in the x-y plane.\hfill\textbf{3}\\
\textbf{(c)} You can also use the trigonometric identities to solve those two equations for t 
to get y as a function of x. Plot that function in the respective limits of x and y.
And check whether the plot is equivalent to the plot in section-(b).\hfill\textbf{3}

%====================================================================================================
\section*{Problem-3 \hfill \textbf{[10]}}
\textbf{(a)} Integrate the function $f(x) = x^2$ for the range of $x=[-\pi, \pi]$.\hfill\textbf{1}\\
\textbf{(b)} considering $f(x)$ to be periodic, that means $f(x + 2\pi) = f(x)$.
\begin{figure}[h]
    \centering
    \includegraphics[scale=.25]{figs/a0-p3-x^2fourier.jpeg}
\end{figure}

Find the Fourier series of the function. Should you need to evaluate the sine integral? \hfill\textbf{3}\\
\textbf{(c)} Determine the number of terms n in the Fourier series expansion such that the leading-order
error is less than $0.1\%$ of the value of the truncated Taylor series at $x=\pi$.\hfill\textbf{4}\\
\textbf{(d)} Integrate the truncated Fourier series in the same limit of x and determine the
deviation with respect to the integration result at part-(a). \hfill\textbf{2}\\
\textit{[You can use some advanced calculator or write a few lines of code to perform the term-wise summation. 
Just mention how you are doing the calculations.]}

%====================================================================================================
\section*{Problem-4: Calculate Integrals \hfill \textbf{[10]}}
\begin{enumerate}
    \item $\int (2cos2x - sin2x)\ e^{-x}\ dx$
    \item $\int sinx\ sin5x\ cos2x\ dx$
    \item $\int_{0}^{\infty}sinh3x\ e^{-2x}\ dx$
    \item $\int_{0}^{\pi}cos2x\ dx$
    \item $\int_{0}^{\pi}cos^{2}2x\ dx$
\end{enumerate}

%====================================================================================================
\section*{Problem-5: Solve the equations and find the roots \hfill \textbf{[6]}}
\begin{enumerate}
    \item $\int f(x)\ dx = cos4x + sin^{2}2x -1$ find $f(x)$.
    \item $\int f(x)\ dx = cos2x\ e^{-x} + t^5$ \textit{[t in independent of x]} find $f(x)$.
    \item $cos4x + sin^{2}2x-1 = 0$ find roots.
\end{enumerate}
%====================================================================================================
\section*{Problem-6:  \hfill \textbf{[4]}}
Prove the relation:
\begin{equation*}
    cos\omega t  + cos(\omega t- \phi) + cos(\omega t- 2\phi) +...+ cos(\omega t- (n-1)\phi) = 
    \frac{sin(\frac{n\phi}{2})}{sin(\frac{\phi}{2})}cos(\omega t -\frac{1}{2}(n-1)\phi)
\end{equation*}\\
\textit{Hint: Try to use complex definition of $cos\theta$, rearrange the terms and use geometric series formula}.


\end{document}
