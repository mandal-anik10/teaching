\documentclass[12pt, a4paper]{article}
\usepackage[utf8]{inputenc}
\usepackage{ragged2e}

\usepackage{graphicx, geometry, hyperref, wrapfig}
\usepackage[dvipsnames]{xcolor}

\definecolor{silver}{RGB}{200,200,200}
\hypersetup{colorlinks=true, linkcolor=RoyalBlue, urlcolor=RoyalBlue}

 \geometry{
 a4paper,
 total={175mm,257mm},
 left=20mm,
 top=15mm,
 }

\usepackage{xcolor} % for defining colour
\usepackage{titlesec} % for customizing sections

% \usepackage{times}        % Use Times New Roman font
% \usepackage{helvet}       % Use Helvetica font
\usepackage{palatino}

\usepackage[T1]{fontenc}

\setlength\parindent{0pt}

% %%%%%%%%%%%%%%%%%%%%%%%%%%%%%%%%%%%%%%%%%%%%%%%%%%%%%%%%%%%%%%
\titleformat{\section}
{\color{UM_DarkBlue}\normalfont\large\bfseries}
{\color{UM_DarkBlue}\thesection}{1em}{}

%%%%%%%%%%%%%%%%%%%%%%%%%%%%%%%%%%%%%%%%%%%%%%%%%%%%%%%%%%%%%%%
\definecolor{UM_Brown}{HTML}{3D190D}
\definecolor{UM_DarkBlue}{HTML}{2264B0}
\definecolor{UM_LightBlue}{HTML}{1CA9E1}
\definecolor{UM_Orange}{HTML}{fEB415}

%%%%%%%%%%%%%%%%%%%%%%%%%%%%%%%%%%%%%%%%%%%%%%%%%%%%%%%%%%%%%%%%

\newcommand{\eg}{{\it e.g.}}
\newcommand{\ie}{{\it i.e.}}

% %%%%%%%%%%%%%%%%%%%%%%%%%%%%%%%%%%%%%%%%%%%%%%%%%%%%%%%%%%%%%%
% \hypersetup{
%     draft=false,
%     final=true,
%     colorlinks=true,
%     citecolor=UM_DarkBlue,
%     anchorcolor=yellow,
%     linkcolor=UM_DarkBlue,
%     urlcolor=UM_DarkBlue,
%     filecolor=green,      
%     pdfpagemode=FullScreen,
%     bookmarksopen=false
%     }
\usepackage{amsmath,amsfonts,amssymb,bm}

%%%%%%%%%%%%%%%%%%%%%%%%%%%%%%%%%%%%%%%%%%%%%%%%%%%%%%%%%%%%%
% Sets and Notations
\newcommand{\reals}{\mathbb{R}}
\newcommand{\integers}{\mathbb{Z}}

%%%%%%%%%%%%%%%%%%%%%%%%%%%%%%%%%%%%%%%%%%%%%%%%%%%%%%%%%%%%%
% Vectors and Matrices
% \renewcommand{\vec}[1]{\bm{\mathrm{#1}}}
\newcommand{\dotp}{\,\boldsymbol{\cdot}\,}
\newcommand{\grad}[1]{\vec{\nabla}#1}
\renewcommand{\div}[1]{\vec{\nabla}\!\dotp\!\vec{#1}}
\newcommand{\curl}[1]{\vec{\nabla}\!\times\!\vec{#1}}



%%%%%%%%%%%%%%%%%%%%%%%%%%%%%%%%%%%%%%%%%%%%%%%%%%%%%%%%%%%%%
% Derivatives
\newcommand{\dv}[2]{\frac{d#1}{d#2}}
\newcommand{\ndv}[3][2]{\frac{d^{\,#1}#2}{d#3^{\,#1}}}

\newcommand{\pdv}[2]{\frac{\partial#1}{\partial#2}}
\newcommand{\npdv}[3][2]{\frac{\partial^{\,#1}#2}{\partial#3^{#1}}}
 
\title{OWO-GAship}
\author{Anik Mandal}
\date{January 2025}
\pagenumbering{arabic}

%====================================================================================================
\begin{document}

\begin{minipage}[t][][c]{0.1\textwidth}
    \begin{flushleft}
        \includegraphics[height=2.5cm]{tex-resources/Ashoka Logo.png}
    \end{flushleft}
\end{minipage}
\begin{minipage}[t][][c]{0.85\textwidth}
    \begin{center}
        {\LARGE Oscillations, Wave and Optics}\\ \vspace{0.5em}
        \textsc{(Spring 2025)}\\
        \vspace{1em}
        \textbf{\Large ASSIGNMENT-6} \\
    \end{center}
\end{minipage}
\vspace{10pt}\\
\rule[0em]{\textwidth}{0.75pt}

\flushleft{Topics: Fresnel Relations}\hfill 
Total Marks: 40   \\
\flushleft{Date: 11th Apr, 2025}\hfill
\fbox{\textbf{\large 
Due: 20th Apr, 2025} (EoD)}\\
\vspace{.2cm}
\rule[0em]{\textwidth}{1.75pt}
\vspace{-1cm}
%====================================================================================================
%====================================================================================================
\justifying

\section*{Problem \hfill \textbf{[40]}}
Consider a polarised EMW, travelling in a medium with refractive index $n_1$ insidents 
on a medium interface at $z=0$ at angle $\theta_i$ as shown in the image. Some fraction of the EMW 
get reflected back to the same medium but some fraction get refracted into a medium of refractive 
index $n_2$.($n_2 > n_1$)
\begin{figure*}[h]
    \centering
    \includegraphics*[scale=0.4]{figs/EMW_polarization.png}
\end{figure*}

\noindent
Consider the polarization state in which the electric components of the incident, reflected, and 
refracted waves are all parallel to the interface. \\
\textbf{1.} Write down the governing Maxwell's equation for the given polarization state.\\
\textbf{2.} Assume possible solutions and using the boundary condition at the interface, derive
the Fresnel relations for the polarization in which the electric field is parallel to the interface.\\
\textbf{3.} Using any graphing calculator draw the reflection and transmission coefficient for 
incident angle running from $0^{\circ}$ to $90^{\circ}$.\\
\textbf{4.} Discuss the total internal reflection in details when $n_1>n_2$ and incident angle is 
more than critical angle for the same state of polarised light. 


\end{document}
